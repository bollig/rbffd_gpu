\chapter{Installation}

This chapter shows how {\ViennaCL} can be integrated into a project and how the
examples are built. The necessary steps are outlined for several different
platforms, but we could not check every possible combination of hardware,
operating system and compiler. If you experience any trouble, please write to
the maining list at \\
\begin{center}
\texttt{viennacl-support$@$lists.sourceforge.net} 
\end{center}


% -----------------------------------------------------------------------------
% -----------------------------------------------------------------------------
\section{Dependencies}
% -----------------------------------------------------------------------------
% -----------------------------------------------------------------------------
\label{dependencies}
{\ViennaCL} uses the {\CMake} build system for multi-platform support.
Thus, before you proceed with the installation of {\ViennaCL}, make sure you
have a recent version of {\CMake} installed.

To use {\ViennaCL}, the following prerequisites have to be fulfilled:
\begin{itemize}
 \item A recent C++ compiler (e.g.~{\GCC} version 4.2.x or above and Visual C++
2008 are known to work)
 \item {\OpenCL}~\cite{khronoscl,nvidiacl} for accessing compute devices (GPUs);
see Section~\ref{opencllibs} for details.
(optional, since iterative solvers can also be used standalone with other libraries (\ublas, Eigen, MTL4))
\end{itemize}


The full potential of {\ViennaCL} is only available with the following optional libraries:
\begin{itemize}
 \item {\CMake}~\cite{cmake} as build system (optional, but highly recommended for building examples)
 \item {\ublas} (shipped with {\Boost}~\cite{boost}) provides the same interface as {\ViennaCL} and allows to switch between CPU and GPU seamlessly, see the tutorials.
 \item Eigen \cite{eigen} can be used to fill {\ViennaCL} types directly. Moreover, the iterative solvers in {\ViennaCL} can directly be used with Eigen objects.
 \item MTL 4 \cite{mtl4} can be used to fill {\ViennaCL} types directly. Even though MTL 4 provides its own iterative solvers, the {\ViennaCL} solvers can also be used with MTL 4 objects.
\end{itemize}

%The use of {\OpenMP} for the benchmark suite allows fair comparisons between your multi-core CPU and your compute device (e.g.~GPU).

\section{Generic Installation of ViennaCL} \label{sec:viennacl-installation}
Since {\ViennaCL} is essentially a header-only library (the only exception is
described in Chapter \ref{chap:tuning}), it is sufficient to copy the folder
\lstinline|viennacl/| either into your project folder or to your global system
include path. On Unix based systems, this is often \lstinline|/usr/include/| or
\lstinline|/usr/local/include/|. If the OpenCL headers are not installed on your system,
you should repeat the above procedure with the folder \lstinline|CL/|.

On Windows, the situation strongly depends on your development environment. We advise users
to consult the documentation of their compiler on how to set the include
path correctly. With Visual Studio this is usually something like \texttt{C:$\setminus$Program Files$\setminus$Microsoft Visual Studio 9.0$\setminus$VC$\setminus$include}
and can be set in \texttt{Tools -> Options -> Projects and Solutions -> VC++-\-Directories}. The include and library directories of your {\OpenCL} SDK should also be added there.

\NOTE{If multiple {\OpenCL} libraries are available on the host system, one has
to ensure that the intended one is used.}


% -----------------------------------------------------------------------------
% -----------------------------------------------------------------------------
\section{Get the {\OpenCL} Library}
\label{opencllibs}
% -----------------------------------------------------------------------------
% -----------------------------------------------------------------------------
In order to compile and run {\OpenCL} applications, a corresponding library
(e.g.~\texttt{libOpenCL.so} under Unix based systems) and is required.
If {\OpenCL} is to be used with GPUs, suitable drivers have to be installed. This section describes how these can be acquired.

\TIP{Note, that for Mac OS X systems there is no need to install an {\OpenCL} 
capable driver and the corresponding library. 
The {\OpenCL} library is already present if a suitable graphics 
card is present. The setup of {\ViennaCL} on Mac OS X is discussed in
Section~\ref{apple}.}

\subsection{\NVIDIA Driver}
\NVIDIA provides the {\OpenCL} library with the GPU driver. Therefore, if a 
\NVIDIA driver is present on the system, the library is too. However, 
not all of the released drivers contain the {\OpenCL} library. 
A driver which is known to support {\OpenCL}, and hence providing the required
library, is $260.19.21$. Note that the latest {\NVIDIA} drivers do not include
the {\OpenCL} headers anymore. Therefore, the official {\OpenCL} headers from
the Khronos group \cite{khronoscl} are also shipped with {\ViennaCL} in the
folder \lstinline|CL/|.

\subsection{AMD Accelerated Parallel Processing SDK (formerly Stream SDK)} \label{sec:opencl-on-ati}
AMD provides the {\OpenCL} library with the Accelerated Parallel Processing (APP)
SDK~\cite{atistream}. At the release of {\ViennaCLversion}, the latest version of the
SDK is $2.4$. If used with AMD GPUs, recent AMD GPU drivers are typically required. If {\ViennaCL} is to be run on multi-core CPUs,
no additional GPU driver is required. The installation notes
of the APP SDK provides guidance throughout the
installation process~\cite{atistreamdocu}. 

\TIP{If the SDK is installed in a non-system wide location on UNIX-based systems, be sure to add the
{\OpenCL} library path to the \texttt{LD\_LIBRARY\_PATH} environment variable.
Otherwise, linker errors will occur as the required library cannot be found.}

It is important to note that the AMD APP SDK does not provide \OpenCL
certified double precision support~\cite{atidouble} on some CPUs and GPUs. In
\ViennaCL 1.0.x, double precision was only experimentally available in
{\ViennaCL} by defining one of the preprocessor constants
\begin{lstlisting}
// for CPUs:
#define VIENNACL_EXPERIMENTAL_DOUBLE_PRECISION_WITH_STREAM_SDK_ON_CPU
// for GPUs:
#define VIENNACL_EXPERIMENTAL_DOUBLE_PRECISION_WITH_STREAM_SDK_ON_GPU
\end{lstlisting}
prior to any inclusion of {\ViennaCL} header files. With
{\ViennaCLminorversion}, this is not necessary anymore and double precision
support is enabled by default -- provided that it is available on the device.

\NOTE{The functions \texttt{norm\_1}, \texttt{norm\_2}, \texttt{norm\_inf} and
\texttt{index\_norm\_inf} are known to cause problems on GPUs in
double precision using ATI Stream SDK v2.1.}

\subsection{INTEL OpenCL SDK} \label{sec:opencl-on-intel}
At the time of this release, a beta-version of an {\OpenCL} SDK by INTEL is available.

Even though the SDK is still in beta-state, {\ViennaCL} works fine with the INTEL OpenCL SDK on Windows and Linux.
The correct linker path is set automatically in \lstinline|CMakeLists.txt| when using the {\CMake} build system, cf.~Sec.~\ref{sec:viennacl-installation}.


% -----------------------------------------------------------------------------
% -----------------------------------------------------------------------------
\section{Building the Examples and Tutorials}
% -----------------------------------------------------------------------------
% -----------------------------------------------------------------------------
For building the examples, we suppose that {\CMake} is properly set up
on your system. The other dependencies are listed in Tab.~\ref{tab:tutorial-dependencies}.

\begin{table}[tb]
\begin{center}
\begin{tabular}{l|l}
Tutorial No. & Dependencies\\
\hline
\texttt{tutorial/blas1.cpp}      & {\OpenCL} \\
\texttt{tutorial/blas2.cpp}      & {\OpenCL}, {\ublas} \\
\texttt{tutorial/blas3.cpp}      & {\OpenCL}, {\ublas} \\
\texttt{tutorial/iterative.cpp}  & {\OpenCL}, {\ublas} \\
\texttt{tutorial/iterative-ublas.cpp}        & {\ublas}  \\
\texttt{tutorial/iterative-eigen.cpp}        & {\Eigen}   \\
\texttt{tutorial/iterative-mtl4.cpp}         & {\MTL}    \\
\texttt{tutorial/custom-kernel.cpp}          & {\OpenCL} \\
\texttt{tutorial/custom-context.cpp}         & {\OpenCL} \\
\texttt{tutorial/eigen-with-viennacl.cpp}    & {\OpenCL}, {\Eigen} \\
\texttt{tutorial/mtl4-with-viennacl.cpp}     & {\OpenCL}, {\MTL} \\
\texttt{tutorial/viennacl-info.cpp}          & {\OpenCL} \\
\texttt{benchmarks/vector.cpp}  & {\OpenCL} \\
\texttt{benchmarks/sparse.cpp}  & {\OpenCL}, {\ublas} \\
\texttt{benchmarks/solver.cpp}  & {\OpenCL}, {\ublas} \\
\texttt{benchmarks/opencl.cpp}  & {\OpenCL} \\
\texttt{benchmarks/blas3.cpp}   & {\OpenCL} \\
\end{tabular}
\caption{Dependencies for the examples in the \texttt{examples/} folder}
\label{tab:tutorial-dependencies}
\end{center}
\end{table}

Before building the examples, customize \texttt{CMakeLists.txt} in the {\ViennaCL} root folder for your needs.
Per default, all examples using {\ublas}, {Eigen} and {MTL4} are turned off.
Please enable the respective examples based on the libraries available on your machine.
Directions on how to accomplish this are given directly within the \texttt{CMakeLists.txt} file.

\subsection{Linux}
To build the examples, open a terminal and change to:
\begin{lstlisting}
 $> cd /your-ViennaCL-path/build/
\end{lstlisting}
Execute
\begin{lstlisting}
 $> cmake ..
\end{lstlisting}
to obtain a Makefile and type
\begin{lstlisting}
 $> make 
\end{lstlisting}
to build the examples. If some of the dependencies in Tab.~\ref{tab:tutorial-dependencies} are not fulfilled, you can build each example separately:
\begin{lstlisting}
 $> make blas1             #builds the blas level 1 tutorial
 $> make vectorbench       #builds vector benchmarks
\end{lstlisting}

\TIP{Speed up the building process by using jobs, e.g. \keyword{make -j4}.}

\subsection{Mac OS X}
\label{apple}
The tools mentioned in Section \ref{dependencies} are available on 
Macintosh platforms too. 
For the {\GCC} compiler the Xcode~\cite{xcode} package has to be installed.
To install {\CMake} and {\Boost} external portation tools have to be used, 
for example, Fink~\cite{fink}, DarwinPorts~\cite{darwinports} 
or MacPorts~\cite{macports}. Such portation tools provide the 
aforementioned packages, {\CMake} and {\Boost}, for macintosh platforms. 

\TIP{If the {\CMake} build system has problems detecting your {\Boost} libraries, 
determine the location of your {\Boost} folder. 
Open the \texttt{CMakeLists.txt} file in the root directory of {\ViennaCL} and 
add your {\Boost} path after the following entry: 
\texttt{IF(\${CMAKE\_SYSTEM\_NAME} MATCHES "Darwin")} }

The build process of {\ViennaCL} on Mac OS is similar to Linux.

\subsection{Windows}
In the following the procedure is outlined for \texttt{Visual Studio}: Assuming that an {\OpenCL} SDK and {\CMake} is already installed, Visual Studio solution and project files can be created using {\CMake}:
\begin{itemize}
\item Open the {\CMake} GUI.
\item Set the {\ViennaCL} base directory as source directory.
\item Set the \texttt{build/} directory as build directory.
\item Click on 'Configure' and select the appropriate generator (e.g.~\texttt{Visual Studio 9 2008})
\item Click on 'Generate' (you may need to click on 'Configure' one more time before you can click on 'Generate')
\item The project files can now be found in the {\ViennaCL} build directory, where they can be opened and compiled with Visual Studio (provided that the include and library paths are set correctly, see Sec.~\ref{sec:viennacl-installation}).
\end{itemize}

\TIP{The examples and tutorials should be executed from within the \texttt{build/} directory of {\ViennaCL}, otherwise the sample data files cannot be found.}






