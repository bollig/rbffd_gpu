
\chapter*{Versioning}  \addcontentsline{toc}{chapter}{Versioning}

Each release of {\ViennaCL} carries a three-fold version number, given by\\
\begin{center}
 \texttt{ViennaCL X.Y.Z} . \\
\end{center}
For users migrating from an older release of {\ViennaCL} to a new one, the
following guidelines apply:
\begin{itemize}
 \item \texttt{X} is the \emph{major version number}, starting with \texttt{1}.
A change in the major version number is not necessarily API-compatible with any
versions of ViennaCL carrying a different major version number. In particular,
end users of {\ViennaCL} have to expect considerable code changes when changing
between different major versions of {\ViennaCL}.

 \item \texttt{Y} denotes the \emph{minor version number}, restarting with zero
whenever the major version number changes. The minor version number is
incremented whenever significant functionality is added to {\ViennaCL}.
The API of an older release of {\ViennaCL} with smaller minor version number
(but same major version number) is \emph{essentially} compatible to the new
version, hence end users of {\ViennaCL} usually do not have to alter their
application code, unless they have used a certain functionality that was not
intended to be used and removed in the new version.

 \item \texttt{Z} is the \emph{revision number}. If either the major or the
minor version number changes, the revision number is reset to zero. Releases of
{\ViennaCL}, that only differ in their revision number, are API compatible.
Typically, the revision number is increased whenever bugfixes are applied,
compute kernels are improved or some extra, not significant functionality is
added.
\end{itemize}

\TIP{Always try to use the latest version of {\ViennaCL} before submitting bug
reports!}
